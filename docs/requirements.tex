\documentclass[12pt]{article}
\usepackage[utf8]{inputenc}
\usepackage[brazil]{babel}
\usepackage{graphicx}
\usepackage{geometry}
\usepackage{longtable}
\usepackage{hyperref}

\hypersetup{
    colorlinks,
    citecolor=black,
    filecolor=black,
    linkcolor=black,
    urlcolor=black
}
\geometry{a4paper, margin=2.5cm}


\title{Levantamento de Requisitos\\Projeto do Sistema de Gerenciamento de Alimentação da Hestia}
\date{\today}

\begin{document}

\maketitle

\tableofcontents
\newpage

\section{Introdução}

Este documento tem como objetivo descrever os requisitos do projeto do sistema de monitoramento e gerenciamento de alimentação da Hestia.

\section{Objetivos do Projeto}

Os objetivos do projeto podem ser categorizados em três marcos principais:

\begin{itemize}
    \item \textbf{Primeira Etapa (Mundial):}
    \begin{itemize}
        \item Desenvolver um sistema básico de monitoramento de tensão, corrente e temperatura das baterias.
        \item Implementar comunicação inicial com o sistema principal (uRos)
        \item Implementar um sistema de proteção contra curto-circuito e sobrecarga.
        \item Implementar um sistema para possibilitar o chaveamento de célular individuais por um switch manual.
        \item Desenvolver uma case para as baterias e suas placas de circuito.
        \item Confeccionar as placas de circuito impresso através do processo de corrosão.
    \end{itemize}
    \item \textbf{Segunda Etapa (Nacional):}
    \begin{itemize}
        \item Implementar sistema de chaveamento por software.
        \item Adicionar um sistema de monitoramento de carga das baterias.
        \item Confeccionar as placas de circuito impresso fabricadas por uma empresa terceira.
        \item Desenvolver um sistema terceiro para carregar as baterias de maneira eficiente.
    \end{itemize}
    \item \textbf{Terceira Etapa (Longo Prazo):}
    \begin{itemize}
        \item Otimizar o consumo de energia do sistema.
        \item Refinar o sistema de modo geral.
        \item Implementar um gerenciamento inteligente das baterias.
        \item Projetar e contruir uma nova bateria
    \end{itemize}

\end{itemize}

\section{Requisitos Funcionais}

\begin{longtable}{|p{0.1\textwidth}|p{0.8\textwidth}|}
\hline
\textbf{ID} & \textbf{Descrição} \\
\hline
RF01 & O sistema deve suportar pelo menos 4 baterias de litio em paralelo. \\
\hline
RF02 & O sistema deve medir a corrente de uma bateria individual. \\
\hline
RF03 & O sistema deve medir a tensão de uma bateria individual. \\
\hline
RF04 & O sistema deve medir a temperatura de uma bateria individual. \\
\hline
RF05 & O sistema deve permitir o desligamento manual de uma bateria via uma chave no circuito. \\
\hline
RF06 & O sistema deve comunicar as leituras ao controlador do robô via UART ou CAN. \\
\hline
RF07 & O sistema deve possuir um sistema de proteção contra curto-circuito e sobrecarga. \\
\hline
RF08 & O sistema deve possuir um sistema de proteção contra picos de corrente e corrente reversa. \\
\hilne
RF09 & O sistema deve ser capaz de fornecer corrente o suficiente para o funcionamento do robô
\hline
RF010 & O sistema deve fornecer ao robo uma autonomia mínima de 30 min ao robô
\hline
RF011 & O sistema deve permitir realizar a troca de baterias sem que seja necessário mexer na fiação 
\hline
\end{longtable}

\section{Requisitos Não Funcionais}

\begin{longtable}{|p{0.1\textwidth}|p{0.8\textwidth}|}
\hline
\textbf{ID} & \textbf{Descrição} \\
\hline
RNF01 & O firmware deve ser escrito em C ou C++ com suporte a atualização futura. \\
\hline
RNF02 & Os esquemáticos devem ser feitos no software KiCad. \\
\hline

\end{longtable}

\section{Restrições}

\begin{itemize}
    \item O sistema deve ocupar espaço máximo de 100x80mm (temporario depois colocar mais).

\end{itemize}

\section{Interfaces Esperadas}

\begin{itemize}
    \item Comunicação via UART, I2C ou CAN.
    \item Interface visual (opcional) via display OLED I2C.
    \item Entrada de alimentação dedicada de 5V para o microcontrolador.
\end{itemize}

\section{Diagrama de Blocos do Sistema}

\vspace{1cm}
\begin{center}
    \end{center}
\vspace{1cm}

\section{Tabela de Especificações Iniciais}

\begin{longtable}{|p{0.3\textwidth}|p{0.6\textwidth}|}
\hline
\textbf{Parâmetro} & \textbf{Valor} \\
\hline
Número de baterias & Ao menos 4 \\
\hline
Tensão nominal por bateria & 36V \\
\hline
Tensão total máxima & 36V \\
\hline
Microcontrolador & ESP32 \\
\hline
Sensor de corrente & (colocar CI) ou similar \\
\hline
Sensores de temperatura & (colocar CI) ou similares \\
\hline
\end{longtable}


\end{document}
